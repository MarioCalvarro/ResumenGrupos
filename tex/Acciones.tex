\chapter{Acción de un grupo\\ sobre un conjunto}
\section{Ecuación de clases}
\subsection{Acciones, órbitas y estabilizadores}

\begin{defi}
Denominamos \textbf{acción de un grupo} $G$ sobre un conjunto $X \neq \emptyset$ a cualquier homomorfismo:
\begin{align*}
    G &\rightarrow \mathrm{Biy}\left( X \right)\\ 
    g &\mapsto \tilde{g}
\end{align*}
Esto define una relación de equivalencia tal que $x \sim y \Leftrightarrow \exists g \in G : y = \tilde{g}\left( x \right)$.

La clase de equivalencia definida así se denomina $\mathbf{G}$\textbf{-órbita} de $x$ bajo la acción de $G$, $O_{G, x} := \left\{ \tilde{g}\left( x \right) : g \in G \right\}$.
\end{defi}
\begin{obs}
$\left\{ O_x : x \in X \right\}$ particiona $X$ y, siendo $R \subset X$ un conjunto de representantes de las clases, se cumple $X = \bigsqcup_{x \in R} O_x$. Por tanto, $\card\left( X \right) = \sum_{x \in R} \card\left( O_x \right)$. 
\end{obs}

\begin{defi}
Llamamos \textbf{estabilizador} de $x \in X$ bajo la acción de $G$ al subgrupo:
\[
\mathrm{Stab}_G \left( x \right) := \left\{ g \in G : \tilde{g}\left( x \right) = x \right\}
\]
\end{defi}

\subsection{Órbitas y estabilizadores}
\begin{prop}[Cardinal de una órbita]
Si $G$ actúa sobre $X$ y $x \in X$, se cumple $\card\left( O_x \right) = \left[ G : \mathrm{Stab}_G\left( x \right) \right]$.
\end{prop}

\begin{coro}[Fórmula de las órbitas]
Sea $R$ conjunto de representantes de las órbitas de $X$, finito, bajo la acción de $G$. Entonces,
\[
\card\left( X \right) = \sum_{x \in R} \left[ G : \mathrm{Stab}_G\left( x \right) \right] 
\]
\end{coro}

\subsection{Aplicaciones a los $p$-grupos}
\begin{defi}
Llamamos $\mathbf{p}$\textbf{-grupo} a aquellos cuyo orden es potencia de un número primo $p$.
\end{defi}
\begin{lema}[Centro de un $p$-grupo]
Sea $H \neq \left\{ 1_G \right\} \le G$, $p$-grupo. Entonces, $H \cap \mathcal{Z}\left( G \right) \neq \left\{ 1_G \right\}$. En particular, $\mathcal{Z}\left( G \right) \neq \left\{ 1_G \right\}$, por lo que $G$ no es simple salvo si $\ord\left( G \right) = p$.
\end{lema}

\begin{lema}[Criterio de abelianidad]
\begin{itemize}
\item Sean $p$, nº primo, $n \in \mathbb{N}$ y $G: \ord\left( G \right) = p^n$. Entonces, $\ord \left( \mathcal{Z}\left( G \right) \right) \neq p^{n-1}$. En particular, si $\ord\left( G \right) = p^3$, no abeliano, entonces $\ord\left( \mathcal{Z}\left( G \right) \right) = p$.

\item Todo $G$ de orden $p^2$, es abeliano.
\end{itemize} 
\end{lema}

\begin{lema}
Sean $p$, nº primo, $G$ finito y $H \le G$ que es $p$-grupo. Entonces $\left[ G : H \right] \equiv \left[ N_G\left( H \right) : H \right] \bmod p$.
\end{lema}

\section{Teorema de Cauchy}
\begin{theo}[de Cauchy]
Sea $p$, nº primo, y $G$ grupo de orden múltiplo de $p$. Entonces, el nº de subgrupos de $G$ de orden $p$ es congruente con $1 \bmod p$. En particular, \underline{$\exists a \in G$ de orden $p$}. 
\end{theo}
