\chapter{Grupos de permutaciones}
\section{Generalidades}
\subsection{Grupo simétrico}
Siendo $n \in \mathbb{N}$, denotamos $I_n := \left\{ x \in \mathbb{Z} : 1 \le x \le n \right\}$ y $\mathcal{S}_n$ al conjunto de biyecciones de $I_n$ en si mismo, que tiene $\card\left( \mathcal{S}_n \right) = n!$. Forma el llamado \textbf{grupo de permutaciones} con la composición ``al revés''.
\begin{align*}
    \sigma \cdot \tau := \sigma \tau : I_n &\rightarrow I_n\\
    j &\mapsto \tau\left( \sigma \left( j \right) \right)
\end{align*}

\begin{theo}[de Cayley]
Todo grupo $G$ es isomorfo a un subgrupo de $\mathrm{Biy}\left( G \right)$. En particular, 
todo grupo finito es isomorfo a un subgrupo del grupo de permutaciones.
\end{theo}

\begin{defi}[Soporte]
Llamamos \textbf{soporte} de una permutación $\sigma \in \mathcal{S}_n$ al conjunto:
\[
\mathrm{sop}\left( \sigma \right) := \left\{ j \in I_n: \sigma\left( j \right) \neq j \right\}
\]
y decimos que dos permutaciones son \textbf{disjuntas} si lo son sus soportes.
\end{defi}

\begin{prop}
\begin{itemize}
    \item Sea $j \in \mathrm{sop}\left( j \right)$, entonces $\sigma\left( j \right) \in \mathrm{sop}\left( j \right)$.
    \item Dos permutaciones disjuntas conmutan.
\end{itemize}
\end{prop}

\begin{defi}[Ciclos]
Una permutación $\sigma \in \mathcal{S}_n$ se denomina \textbf{ciclo de longitud} $k$ si $\exists i_1, \ldots, i_k \in I_n$ tales que $\mathrm{sop}\left( \sigma \right) = \left\{ i_1, \ldots, i_k \right\}$ y
\[
    \sigma\left( i_1 \right) = i_2,\ \sigma\left( i_2 \right) = i_3, \ldots, \sigma\left( i_{k-1} \right) = i_k\ \& \sigma\left( i_k \right) = i_1
\]
Si es de longitud $2$ lo denominaremos como \textbf{transposición}.
\end{defi}

\begin{prop}
\begin{itemize}
    \item Toda permutación es composición de ciclos disjuntos y esta factorización es única salvo el orden de los factores.
    \item Si $\sigma_1, \ldots, \sigma_r$ son ciclos disjuntos y $\mathrm{long}\left( \sigma_i \right) \le \mathrm{long}\left( \sigma_{r+1} \right),\ \forall 1 \le i \le r-1$, se llama \textbf{estructura cíclica} de $\sigma := \sigma_1 \cdots \sigma_r$ a la $r$-tupla $\left( \mathrm{long}\left( \sigma_1 \right), \ldots, \mathrm{long}\left( \sigma_r \right) \right)$.
\end{itemize}
\end{prop}

\begin{lema}
Siendo $\sigma, \tau \in \mathcal{S}_n$ disjuntas tal que $o\left( \sigma \right) = \ell$ y $o\left( \tau \right) = m$, entonces $o\left( \sigma \tau \right) = \mcm\left( \ell, m \right)$
\end{lema}

\begin{coro}
Sea $\sigma := \sigma_1 \cdots \sigma_k$ una factorización en ciclos de la permutación. Entonces,
\[
o\left( \sigma \right) = \mcm\left( \mathrm{long}\left( \sigma_1 \right), \ldots, \mathrm{long}\left( \sigma_k \right) \right)
\]
\end{coro}

\begin{defi}[Índice]
\begin{itemize}
\item $\forall \sigma \in \mathcal{S}_n$ consideramos el endomorfismo, $f_{\sigma} : \mathbb{R}^n \rightarrow \mathbb{R}^n$ que cumple que $f_{\sigma}\left( e_j \right) = e_{\sigma^{-1}\left( j \right)}$. Entonces,
\begin{align*}
    \psi: \mathcal{S}_n &\rightarrow \mathrm{Aut}\left( \mathbb{R}^n \right)\\
    \sigma &\mapsto f_{\sigma}
\end{align*}
es homomorfismo de grupos.

La matriz de $f_{\sigma}$ proviene de desordenar las columnas de la identidad, por tanto, $\det \left( f_{\sigma} \right) \in \left\{ 1, -1 \right\}$. Definimos, pues, el homomorfismo \textbf{índice}:
\[
    \varepsilon := \mathrm{det} \circ \psi: \mathcal{S}_n \rightarrow \mathcal{U}_2
\]

\item Al kernel de $\varepsilon$ se le denota $\mathcal{A}_n$, \textbf{$\mathbf{n}$-ésimo grupo alternado}. Si $\sigma \in \mathcal{A}_n$ se dice \textbf{par} y en caso contrario \textbf{impar}.
\end{itemize} 
\end{defi}

\begin{lema}
Las transposiciones constituyen un sistema generador de $\mathcal{S}_n$.
\end{lema}

\begin{prop}
El ciclo $\sigma := \left( a_1, \ldots, a_k \right) \in \mathcal{S}_n \in \mathcal{A}_n \Leftrightarrow k$ impar.
\end{prop}

\begin{prop}[Sistemas generadores de $\mathcal{S}_n$ y $\mathcal{A}_n$]
\begin{itemize}
\item $\mathcal{S}_n$ es generado por $\left\{ \alpha_i := \left( 1, i \right): 2 \le i \le n \right\}$.
\item $\mathcal{S}_n$ es generado por $\left\{ \tau_i := \left( i, i+1 \right) : 1 \le i \le n-1 \right\}$.
\item $\mathcal{S}_n$ es generado por $\left( 1, 2 \right)$ y $\left( 1, \ldots, n \right)$.
\item $\mathcal{A}_n$ es generado por $\left\{ \sigma_i := \left( 1, 2, i \right) : 3 \le i \le n \right\}$.
\end{itemize} 
\end{prop}

\section{Teorema de Abel}
%\begin{lema}
%Sea $n \ge 3$ y $N \triangleleft \mathcal{A}_n$ con un $3$-ciclo. Entonces $N = \mathcal{A}_n$.
%\end{lema}
%
%\begin{lema}
%Sea $\tau_1, \ldots, \tau_r$ ciclos disjuntos en $\mathcal{S}_n$, donde $\tau_r := \left( a_1, \ldots, a_k \right)$ y $k \ge 3$. Sean:
%\[
%\sigma = \tau_1 \cdots \tau_r,\quad \beta := \left( a_1, a_2, a_3 \right),\ \alpha := \beta \sigma^{-1} \beta^{-1} \sigma
%\]
%Entonces,
%\begin{itemize}
%\item $\mathrm{sop}\left( \alpha \right) \subsetneq \mathrm{sop}\left( \sigma \right)$
%\item Si $k > 3$, entonces $\alpha \neq id$.
%\item Todo $H \triangleleft \mathcal{A}_n$ que contiene a $\sigma$, también contiene a $\alpha$.
%\end{itemize}
%\end{lema}
\begin{theo}[De Abel]
Si $n \ge 5,\ \mathcal{A}_n$ es simple.
\end{theo}

\begin{coro}
Si $n \ge 5$, entonces $\mathcal{A}_n$ es el único subgrupo normal propio de $\mathcal{S}_n$.
\end{coro}

\begin{defi}
$H \le \mathcal{S}_n$ será \textbf{transitivo} si $\forall \left( i, j \right)$ tal que $1 \le i, j \le n,\ \exists \sigma \in H$ tal que $\sigma \left( i \right) = j$.
\end{defi}

\begin{prop}
Si $p \in \mathbb{Z}$ es primo y $H \le \mathcal{S}_p$ transitivo que contiene una transposición, entonces $H = \mathcal{S}_p$.
\end{prop}
